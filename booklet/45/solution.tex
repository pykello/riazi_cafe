\begin{solution}.
The answer for the 7-digits case is equal to 9 and for the 13-digits case it is equal to 585. In general, for the $6n+1$-digits case the solution is equal to $\frac{2^{6n}-1}{7}$.\\[0.2cm]

First, we try to solve the problem for the 7-digits case. For a number to be divisible by 5, the rightmost digit must be either 0 or 5. Since we are only allowed to have digits $5$ and $7$ in the desired numbers, then their rightmost digit should be equal to $5$. Therefore, all desired numbers can be written as $abcdef5$. In order for the  number to be divisible by 7, the remainder of $abcdef0$ over 7 must be equal to 2. This quantity can be formulated as:

$$
\begin{aligned}
abcdef0 \overset{7}{\equiv}  &f \cdot 10  + e \cdot 10^2  + d \cdot 10^3 + \\
&c \cdot 10^4  + b \cdot 10^5  + a \cdot 10^6 
\end{aligned}
$$

Notice that putting a digit as 7 is equivalent to putting that digit as 0 when we only care about its module over 7. Thus we can interpret the problem as choosing a subset $s$ of  $X = \{ a, b, c, d, e, f\}$ whose assigned values are equal to 5 in a way that $f \cdot 10  + e \cdot 10^2  + d \cdot 10^3 + 
c \cdot 10^4  + b \cdot 10^5  + a \cdot 10^6 \overset{7}{\equiv} 2$ when the values of digits $X \setminus s$ are all zero . Moreover, $5 \cdot 10^i$ module 7 is equal to 1, 3, 2, 6, 4, 5 for $1 \leq i \leq 6$ respectively (notice that each number from 1 to 6 appears exactly once in the sequence). Therefore, the  problem is equivalent to counting the subsets of $\{1, 2, 3, 4, 5, 6\}$ whose sum module 7 is equal to 2.

A simple case analysis shows that there are 9 such subsets:
\begin{itemize}
	\item Such subsets of length 1: $\{2\}$.
	\item Such subsets of length 2: $\{3,6\}$, $\{4,5\}$.
	\item Such subsets of length 3: $\{1,2,6\}$, $\{1,3,5\}$, $\{2,3,4\}$.
	\item Such subsets of length 4: $\{1,4,5,6\}$, $\{2,3,5,6\}$.
	\item Such subsets of length 5: $\{1,2,3,4,6\}$.
\end{itemize}

If we continue computing $5 \cdot 10^i$ module 7 for $8 \leq i  \leq 13$ we obtain the same sequence of numbers (1, 3, 2, 6, 4, 5). That is, to find the answer for the 13-digits case, we need to find the number of multisubsets of a multiset that has two appearances of each digit between 1 and 6.

We prove below that (Lemma ~\ref{lemma:module7}) if $X_n$ is a multiset that contains each digit between $1$ and $6$ exactly $n$ times, then the number of its multisubsets whose sum module 7 is equal to 2 is equal to $
\frac{2^{6n}-1}{7}
$ which for the case of $n=2$ (for the 13-digits case) is equal to 585.

\begin{lemma}\label{lemma:module7}
Let $X_n$ be a multiset that contains each digit between $1$ and $6$ exactly $n$ times. The number of  multisubsets of $X_n$ whose sum of values module 7 is equal to $i$ is equal to:
$$
\left\{ \begin{array}{lr}
\frac{2^{6n}+6}{7} & \text{for } i=0\\
\frac{2^{6n}-1}{7} & \text{for } 1\leq i\leq 6
\end{array} \right\}
$$
\end{lemma}
\begin{proof}
Since 7 is a prime number, each of numbers 1 to 6 has a unique multiplicative inverse in module  7. If we have a multisubset whose sum module 7 is equal to  $p$, by multiplying each of its members by $qp^{-1}$ (module 7) the sum of the values of the resulting multisubset module 7 is equal to $q$. Since 7 is prime, such a multiplication gives a 1-to-1 mapping and thus the resulting multisubset is a multisubset of the original multiset. This implies that the number of multisubsets of $X_n$ whose sum module 7 is equal to $i$ are exactly the same for each $1 \leq i \leq 6$. Next, we use induction to finish the proof:\\[0.2cm]

\noindent \textbf{Base} $(n=1)$: We already discussed that for $X_1 = \{1,2,3,4,5,6\}$, the number of subsets whose sum module $7$ is equal to 2 is  9. Due to what mentioned above the same holds for the number of substes whose sum module 7 is equal to $1$, $3$, $5$, and $6$. Therefore, the number of subsets whose sum is divisible by $7$ is equal to $2^6-9\cdot 6=10$ which is as desired.

\noindent \textbf{Induction step} $(n>1)$: To choose a multisubset of $X_n$ whose sum is divisible by 7, we have 7 choices:
\begin{itemize}
	\item (option 1): Choose a multisubset of $X_{n-1}$ whose sum is divisible by 7  and append it to a multisubset of $X_1$ whose sum is divisible by 7.
	\item (options 2-7 for $1 \leq i \leq 6$) Choose a multisubset of $X_{n-1}$ whose sum module $7$ is equal to $i$ and append it to a multisubset of $X_1$ whose sum module 7 is $7-i$.
\end{itemize}
Therefore, the total number of multisubsets of $X_n$ whose sum is divisible by 7 is: $$6 \cdot 9 \cdot \frac{2^{6n-6} - 1}{7} + 10 \cdot \frac{2^{6n-6}+6}{7} = \frac{54 \cdot 2^{6n-6} - 54 + 10 \cdot 2^{6n-6} + 60}{7} = \frac{2^{6n} + 6}{7}.$$
It follows from our discussions above that for any $1 \leq i \leq 6$, the number of multisubsets of $X_n$ whose sum module $7$ is equal to $i$ is equal to $$\frac{2^{6n} - \frac{2^{6n}+6}{7}}{6} = \frac{2^{6n}-1}{7}.$$
\end{proof}
\end{solution}