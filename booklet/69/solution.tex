\begin{solution}
The answer is 1/49.\\[0.2cm]

We start by noting that after $n \geq 2$ shots, the probability of exactly 0 or $n$ successful shots is 0. We prove below that for any $1 \leq k < n$ the odds that $k$ out of $n$ shots are successful is $1/(n-1)$. We prove this by induction:\\[0.2cm]

\textbf{Base case}: For $n=2$, it is trivial to see that the conditions hold.

\textbf{Induction step}: Assume that the condition holds for $n \geq 2$ shots and any $1 \leq k < n$. That is, if we denote the probability of exactly $k$ goals in $n$ shots by $p(k, n)$, the induction hypothesis states that $p(0, n) = p(n, n) = 0$ and for $0 < k < n$ we have $p(k, n) = 1/(n-1)$. Based on this, we prove that $p(k,n+1)$ = $1/n$ for any $1 \leq k \leq n$. We consider the following three cases:

\begin{itemize}
\item $k=1$: Either we have zero goals in the first $n$ shots (probability $p(0,n) = 0$) and the $(n+1)$'th shot is successful (probability 0) or we have one goal in the first $n$ shots (probability $p(1,n) = 1/(n-1)$) and the $(n+1)$'th shot is unsuccessful (probability $(n-1)/n$). Therefore we have:
$$p(1, n+1) = p(0, n) \cdot 0 + p(1, n) \cdot (n-1)/n = 0 + 1/(n-1) \cdot (n-1)/n = 1/n.$$

\item $k=n$: Either we have $n-1$ goals in the first $n$ shots (probability $p(n-1,n) = 1/(n-1)$) and the $(n+1)$'th shot is successful (probability $(n-1)/n$) or we have $n$ goals in the first $n$ shots (probability 0) and the $(n+1)$'th shot is unsuccessful (probability 0). Therefore we have:

$$p(n, n+1) = p(n-1, n) \cdot (n-1)/n + p(n, n) \cdot n-n/n = 1/(n-1) \cdot (n-1)/n + 0 = 1/n.$$

\item $1 < k < n$: Either we have $k-1$ goals in the first $n$ shots (probability $p(k,n)) = 1/(n-1)$) and the $(n+1)$'th shot is successful (probability $(k-1)/n$) or we have $k$ goals in the first $n$ shots (probability $p(k,n)=1/(n-1)$) and the $(n+1)$'th shot is unsuccessful (probability $(n-k)/n$). Therefore we have:

$$p(k, n+1) = p(k-1, n) \cdot (k-1)/n + p(k, n) \cdot (n-k)/n = 1/(n-1) \cdot (k-1)/n + 1/(n-1) \cdot (n-k)/n = 1/n.$$
\end{itemize}

Therefore, the proof also holds for $n+1$ shots and any $1 \leq k < n$ and thus holds for all natural numbers greater than or equal to 2.
\end{solution}
