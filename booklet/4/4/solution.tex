\begin{solution}
The answer is 3.

First, we put four coins on the left side of the scale and the other four coins on the right side of the scale and compare their weights.

First case: If their weight was equal. This means that the different coin is one of the four coins that are not on the scale. Chance that the number of these coins is one to four. By using the scales twice, we compare coin number one with coins number two and three. By looking at the results, you can see which coin is different.

Second case: one of the two sets of coins is heavier than the other. In this case, assume that the coins outside the scale are numbered one to four, the coins in the lighter category are numbered five to eight, and the coins in the heavy category are numbered nine to twelve. The answer is definitely one of the five to twelve coins.

In this case, we put coins number one, two, three and five on one side of the scale and coins six, seven, nine and ten on the other side of the scale. If the weight of both sides is equal, the answer is one of eight, eleven and twelve coins, which can be found by comparing eleven and twelve in the next step.

If the first category was lighter than the second category, the answer is either five, nine or ten. Similarly, by comparing nine and ten, the answer can be found by using the scale once.

If the second category was lighter, it means that the answer is either a six coin or a seven coin, which can be found with a comparison.

Solution link on Twitter:  \href{https://twitter.com/Riazi_Cafe/status/1666297407208845312}{https://twitter.com/Riazi_Cafe/status/1666297407208845312}\end{solution}