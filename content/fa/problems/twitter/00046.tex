\begin{solution}
مسئله یک بازی حاصل جمع صفر را توصیف می کند. در این بازی، هر بازیکن دارای 6 استراتژی 
$\langle 1, 2, 3 \rangle$, $\langle 1, 3, 2 \rangle$, $\langle 2, 1, 3 \rangle$, $\langle 2, 3, 1 \rangle$, $\langle 3, 1, 2 \rangle$,  و $\langle 3, 2, 1 \rangle$
هست. جدول زیر امتیاز بازیکنان برای هر جفت استراتژی را نشان می دهد:

\begin{center}
\begin{tabular}{|c|c|c|c|c|c|c|}
\hline
& $\langle 1, 2, 3\rangle$& $\langle 1, 3, 2\rangle$& $\langle 2, 1, 3\rangle$& $\langle 2, 3, 1\rangle$& $\langle 3, 1, 2\rangle$& $\langle 3, 2, 1\rangle$\\
\hline
$\langle 1, 2, 3\rangle$& $(0-0)=0$& $(30-40)=-10$& $(30-30)=0$& $(30-70)=-40$& $(60-40)=20$& $(30-40)=-10$\\
\hline
$\langle 1, 3, 2\rangle$& $(40-30)=10$& $(0-0)=0$& $(70-30)=40$& $(30-30)=0$& $(30-40)=-10$& $(30-70)=-40$\\
\hline
$\langle 2, 1, 3\rangle$& $(30-30)=0$& $(30-70)=-40$& $(0-0)=0$& $(30-40)=-10$& $(30-40)=-10$& $(60-40)=20$\\
\hline
$\langle 2, 3, 1\rangle$& $(70-30)=40$& $(30-30)=0$& $(40-30)=10$& $(0-0)=0$& $(30-70)=-40$& $(30-40)=-10$\\
\hline
$\langle 3, 1, 2\rangle$& $(40-60)=-20$& $(40-30)=10$& $(40-30)=10$& $(70-30)=40$& $(0-0)=0$& $(30-30)=0$\\
\hline
$\langle 3, 2, 1\rangle$& $(40-30)=10$& $(70-30)=40$& $(40-60)=-20$& $(40-30)=10$& $(30-30)=0$& $(0-0)=0$\\

\hline
\end{tabular}
\end{center}

از آنجایی که امتیازها برای صفحات وب 1 و 2 یکسان هستند، می توانیم تعداد استراتژی ها را با فرض اینکه ابتدا صفحه وب سوم را در جایگشت قرار می دهیم و سپس به طور تصادفی صفحات وب اول و دوم را در موقعیت های باقی مانده قرار می دهیم به 3 کاهش دهیم. به دلیل تقارن امتیاز استراتژی بهینه با این کاهش فرقی نخواهد کرد. بنابراین، جدول امتیازها را می توان به صورت زیر خلاصه کرد:


\begin{center}
\begin{tabular}{|c|c|c|c|}
\hline
& $\langle 3, *, *\rangle$& $\langle *, 3, *\rangle$& $\langle *, *, 3\rangle$\\
\hline
$\langle 3, *, *\rangle$& $0$& $25$& $-5$\\
\hline
$\langle *, 3, *\rangle$& $-25$& $0$& $25$\\
\hline
$\langle *, *, 3\rangle$& $5$& $-25$& $0$\\
\hline
\end{tabular}
\end{center}

استراتژی $\langle 3، *، *\rangle$ را با احتمال $p$، استراتژی $\langle *، 3، *\rangle$ با احتمال $q$ و استراتژی $\langle *، *، 3 \rangle$ با احتمال $1-p-q$ بازی کنیم.
ما می‌خواهیم در بدترین حالت، امتیاز خود را بیشینه کنیم و بنابراین امتیاز خود را برابر با کمترین امتیاز در مقایل سه بازی حریف به صورت زیر فرموله می‌کنیم:

$$\min\{-25q + 5 (1-p-q)، 25p - 25 (1-p-q)، -5p+25q \}$$

عبارت فوق به ازای  $p=5/11، q=1/11، (1-p-q) = 5/11$ بیشینه می‌شود. بنابراین در استراتژی بهینه، به صورت زیر بازی می کنیم:

\begin{center}
	\begin{tabular}{|c|c|}
		\hline
		استراتژی & احتمال\\
		\hline
		$\langle 1, 2, 3\rangle$& $5/22$\\
		\hline
		$\langle 1, 3, 2\rangle$& $1/22$\\
		\hline
		$\langle 2, 1, 3\rangle$& $5/22$\\
		\hline
		$\langle 2, 3, 1\rangle$& $1/22$\\
		\hline
		$\langle 3, 1, 2\rangle$& $5/22$\\
		\hline
		$\langle 3, 2, 1\rangle$& $5/22$\\
		\hline
	\end{tabular}
\end{center}

\end{solution}