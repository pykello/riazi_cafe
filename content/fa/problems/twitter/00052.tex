\begin{solution}
راه حل برابر با $34$ است.

از این به بعد، وقتی از اصطلاح گراف با $n$ رأس استفاده می‌کنیم، منظور ما گراف بدون وزن و بدون جهت است که رأس‌های آن با اعداد $1, 2, 3, \ldots, n$ برچسب‌گذاری شده‌اند. فرض کنید $G$ یک گراف با $n$ رأس و $P = \langle p_1, p_2, \ldots, p_n \rangle$ یک جایگشت از اعداد $1$ تا $n$ باشد. ما تبدیل گراف $G$ توسط جایگشت $P$ را به عنوان گرافی مشابه $G$ تعریف می‌کنیم با این تفاوت که برچسب هر رأس $i$ با $p_i$ جایگزین می‌شود. این تبدیل را با $T(G, P)$ نشان می‌دهیم.
ما از قضیه زیر برای شمارش تعداد گراف‌های ناهمگون با 5 رأس استفاده می‌کنیم. اثبات این قضیه را به انتهای راه حل موکول می‌کنیم.
\begin{lemma}\label{قضیه}
برای یک $n$ داده شده، فرض کنید $x_n$ تعداد جفت‌های $(G, P)$ باشد به طوری که $G$ یک گراف با $n$ رأس، $P$ یک جایگشت از اعداد $1,2, \ldots, n$، و $T(G, P) = G$ باشد. تعداد گراف‌های ناهمگون با $n$ رأس برابر با $x_n/n!$ است.
\end{lemma}

بر اساس قضیه، پاسخ این سوال برابر با $x_5 / 120$ است. بنابراین، هدف ما یافتن مقدار $x_5$ است.

ما می‌توانیم جایگشت‌های اعداد $1, 2, \ldots, n$ را بر اساس طول دوره‌های گراف جایگشت آن‌ها دسته‌بندی کنیم:
\begin{itemize}
\item 5 دور اندازه 1: $\langle 1, 2, 3, 4 ,5 \rangle$ تنها جایگشتی است که گراف جایگشت آن 5 دور دارد. همه $2^{10}$ گراف با 5 رأس تحت این جایگشت دست نخورده باقی می‌مانند.
\item 3 دور اندازه 1 و 1 دور اندازه 2: 10 جایگشت وجود دارد که گراف جایگشت آن‌ها 3 دور اندازه 1 و 1 دور اندازه 2 دارد. $2^{7}$ گراف با 5 رأس تحت چنین جایگشت‌هایی دست نخورده باقی می‌مانند.
\item 2 دور اندازه 1 و 1 دور اندازه 3: 20 جایگشت وجود دارد که گراف جایگشت آن‌ها 2 دور اندازه 1 و یک دور اندازه 3 دارد. $2^{4}$ گراف با 5 رأس تحت چنین جایگشت‌هایی دست نخورده باقی می‌مانند.
\item 1 دور اندازه 1 و 1 دور اندازه 4: 30 جایگشت وجود دارد که گراف جایگشت آن‌ها 1 دور اندازه 1 و 1 دور اندازه 4 دارد. $2^{3}$ گراف با 5 رأس تحت چنین جایگشت‌هایی دست نخورده باقی می‌مانند.
\item 1 دور اندازه 2 و 1 دور اندازه 3: 20 جایگشت وجود دارد که گراف جایگشت آن‌ها 1 دور اندازه 2 و 1 دور اندازه 3 دارد. $2^{3}$ گراف با 5 رأس تحت چنین جایگشت‌هایی دست نخورده باقی می‌مانند.
\item 1 دور اندازه 1 و 2 دور اندازه 2: 15 جایگشت وجود دارد که گراف جایگشت آن‌ها 1 دور اندازه 1 و دو دور اندازه 2 دارد. $2^{6}$ گراف با 5 رأس تحت چنین جایگشت‌هایی دست نخورده باقی می‌مانند.
\item 1 دور اندازه 5: 24 جایگشت وجود دارد که گراف جایگشت آن‌ها 1 دور اندازه 5 دارد. 4 گراف با 5 رأس تحت چنین جایگشت‌هایی دست نخورده باقی می‌مانند.
\end{itemize}
بنابراین، $x_5 = 2^{10} + 10\cdot 2^7 + 20\cdot 2^4 + 30\cdot 2^3 + 20\cdot 2^3 + 15\cdot 2^6 + 4!\cdot 4 = 4080$. بنابراین، پاسخ برابر با $4080/120 = 34$ است.

\begin{proof}اثبات قضیه. 
یک ابرگراف با $2^{\binom{n}{2}}$ رأس در نظر بگیرید که هر راس متناظر با یک گراف  $n$ رأسی است. برای هر جایگشت $P$، یک یال بین رأس $u$ و رأس $v$ با برچسب $P$ رسم می‌کنیم اگر $T(u) = v$ باشد. تعداد گراف‌های ناهمگون با $n$ رأس برابر است با تعداد مؤلفه‌های متصل این ابرگراف. توجه داشته باشید که درجات تمامی رئوس این ابرگراف برابر با $n!$ است. علاوه بر این، تمام رئوس هر مؤلفه متصل نماینده گراف‌های همگون هستند. این بدان معناست که برای هر مؤلفه متصل، تعداد یال‌ها بین هر جفت رأس یکسان است، و این مقدار برابر است با تعداد حلقه‌های هر رأس درون آن مؤلفه. بنابراین، جمع تعداد حلقه‌های رئوس در هر مؤلفه برابر است با $n!$. این بدان معناست که کل تعداد حلقه‌ها بر روی $n!$ برابر است با تعداد مؤلفه‌های ابرگراف که برابر است با تعداد گراف‌های ناهمگون با اندازه $n$.
\end{proof}
\end{solution}
