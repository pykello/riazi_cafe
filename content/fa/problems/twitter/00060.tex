\begin{solution}
تولد نیکو  یا هفتم اردیبهشت (روز ۳۸ام سال) است یا هشتم اردیبهشت (روز ۳۹ام سال).

در ادامه تحلیل میکنیم که بعد از هر سوال، تحلیل کنیم که چه چیزی در ذهن نیک و نیکو می گذرد.

وقتی نیک از نیکو می پرسد که آیا تولد او زودتر از تولد نیکو است، با توجه به اینکه آنها تولد یکدیگر را نمی دانند، نیکو فقط در صورتی "نه" می گوید که تولدش اول فروردین (اولین روز سال) باشد. در غیر این صورت، پاسخ او "نمی دانم" خواهد بود. بنابراین، بعد از اولین سوال، برای همه روشن است که تولد نیکو اول فروردین نیست.

وقتی نیکو همین سوال را از نیک می پرسد، نیک از قبل می داند که تولد نیکو اول فروردین نیست. بنابراین او "نه" می گوید اگر تولدش یا اول فروردین یا دوم فروردین باشد؛ در غیر این صورت پاسخ او "نمی دانم" خواهد بود. بنابراین، بعد از سوال نیکو، برای همه روشن است که نیکو نه در اول فروردین و نه در دوم فروردین متولد شده است.

دومین باری که نیک از نیکو می پرسد که آیا تولد او زودتر از تولد اوست، نیکو از اطلاعات بالا آگاه است و بنابراین فقط در صورتی "نه" می گوید  که تولدش یا دوم فروردین یا سوم فروردین باشد. در غیر این صورت، او میگوید "نمی دانم". این روند تا آخرین سوال ادامه دارد. جدول زیر دانش مشترک بعد از هر سوال را نشان می دهد:

\begin{center}
	\begin{tabular}{|c|c|c|}
		\hline
		سوال & جواب & دانش مشترک \\
		\hline
		اولین سوال نیک & نمی‌دانم & تولد نیکو بعد از اول فروردین است \\
		\hline 
		اولین سوال نیکو & نمی‌دانم & تولد نیکو بعد از اول فروردین است \\
		& & و تولد نیک بعد از دوم فروردین است \\
		\hline
		دومین سوال نیک & نمی‌دانم & تولد نیکو بعد از سوم فروردین است \\
		& & و تولد نیک بعد از دوم فروردین است \\
		\hline
		دومین سوال نیکو & نمی‌دانم & تولد نیکو بعد از سوم فروردین است \\
		& & و تولد نیک بعد از چهارم فروردین است \\
		\hline
		$\vdots$ & $\vdots$ & $\vdots$ \\
		\hline
		نوزدهمین سوال نیک & نمی‌دانم & تولد نیکو بعد از ششم اردیبهشت است \\
		& & و تولد نیک بعد از پنجم اردیبهشت است \\
		\hline
		نوزدهمین سوال نیکو & نمی‌دانم & تولد نیکو بعد از ششم اردیبهشت است \\
		& & و تولد نیک بعد از هفتم اردیبهشت است \\
		\hline
		بیستمین سوال نیک & نه & تولد نیکو بعد از هشتم اردیبهشت نیست \\
		& & و تولد نیک بعد از هفتم اردیبهشت است \\
		\hline
	\end{tabular}
\end{center}

\end{solution}

